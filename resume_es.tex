% Author : Sourabh Bajaj
% License : MIT
%------------------------

\documentclass[letterpaper,11pt]{article}

\usepackage{latexsym}
\usepackage[empty]{fullpage}
\usepackage{titlesec}
\usepackage{marvosym}
\usepackage[usenames,dvipsnames]{color}
\usepackage{verbatim}
\usepackage{enumitem}
% \usepackage[hidelinks]{hyperref}
\usepackage{hyperref}
\usepackage{fancyhdr}
\usepackage[english]{babel}
\usepackage{tabularx}
\input{glyphtounicode}

\pagestyle{fancy}
\fancyhf{} % clear all header and footer fields
\fancyfoot{}
\renewcommand{\headrulewidth}{0pt}
\renewcommand{\footrulewidth}{0pt}

% Adjust margins
\addtolength{\oddsidemargin}{-0.5in}
\addtolength{\evensidemargin}{-0.5in}
\addtolength{\textwidth}{1in}
\addtolength{\topmargin}{-.5in}
\addtolength{\textheight}{1.0in}

\urlstyle{same}

\raggedbottom
\raggedright
\setlength{\tabcolsep}{0in}

% Sections formatting
\titleformat{\section}{
  \vspace{-4pt}\scshape\raggedright\large
}{}{0em}{}[\color{black}\titlerule \vspace{-5pt}]

% Ensure that generate pdf is machine readable/ATS parsable
\pdfgentounicode=1

%-------------------------
% Custom commands
\newcommand{\resumeItem}[2]{
  \item\small{
    \textbf{#1}{: #2 \vspace{-2pt}}
  }
}

% Just in case someone needs a heading that does not need to be in a list
\newcommand{\resumeHeading}[4]{
    \begin{tabular*}{0.99\textwidth}[t]{l@{\extracolsep{\fill}}r}
      \textbf{#1} & #2 \\
      \textit{\small#3} & \textit{\small #4} \\
    \end{tabular*}\vspace{-5pt}
}

\newcommand{\resumeSubheading}[4]{
  \vspace{-1pt}\item
    \begin{tabular*}{0.97\textwidth}[t]{l@{\extracolsep{\fill}}r}
      \textbf{#1} & #2 \\
      \textit{\small#3} & \textit{\small #4} \\
    \end{tabular*}\vspace{-5pt}
}

\newcommand{\resumeSubSubheading}[2]{
    \begin{tabular*}{0.97\textwidth}{l@{\extracolsep{\fill}}r}
      \textit{\small#1} & \textit{\small #2} \\
    \end{tabular*}\vspace{-5pt}
}

\newcommand{\resumeSubItem}[2]{\resumeItem{#1}{#2}\vspace{-4pt}}

\renewcommand{\labelitemii}{$\circ$}

\newcommand{\resumeSubHeadingListStart}{\begin{itemize}[leftmargin=*]}
\newcommand{\resumeSubHeadingListEnd}{\end{itemize}}
\newcommand{\resumeItemListStart}{\begin{itemize}}
\newcommand{\resumeItemListEnd}{\end{itemize}\vspace{-5pt}}

%-------------------------------------------
%%%%%%  CV STARTS HERE  %%%%%%%%%%%%%%%%%%%%%%%%%%%%


\begin{document}

%----------HEADING-----------------
\begin{tabular*}{\textwidth}{l@{\extracolsep{\fill}}r}
  \textbf{\href{https://github.com/ccarral}{\Large Carlos Carral Cortés}} & Email : \href{mailto:carloscarral13@gmail.com}{carloscarral13@gmail.com}\\
  \href{https://ccarral.github.io}{https://ccarral.github.io} & Tel: +52-722-538-7535 \\
    & Ubicación : Estado de México, Mexico 
\end{tabular*}

\section{Perfil}
Desarrollador full stack, especializado en el back end. De rápido
aprendizaje y abierto a
cualquier lenguaje, framework o herramienta. Interesado en el desarrollo de
aplicaciones rápidas y en el cómputo de bajo nivel.

%-----------EDUCATION-----------------
\section{Educación}
  \resumeSubHeadingListStart
    \resumeSubheading
      {Universidad Autónoma del Estado de México}{Toluca, Estado de México}
      {Ingeniería en Computación}{Agosto 2017 -- Actualidad}
    \resumeSubheading
      {Tecnológico de Monterrey}{Toluca, Estado de México}
      {Bachillerato}{Agosto 2013 -- Agosto 2016}
  \resumeSubHeadingListEnd

%-----------EXPERIENCE-----------------
\section{Experiencia Laboral}
  \resumeSubHeadingListStart
    \resumeSubheading{UAEMex, Secretaría de Planeación y Desarrollo.}{Toluca, Estado de México}
      {DBA y automatización de procesos}{Agosto 2020 -- Marzo 2021}
    \resumeSubheading{Aventura Paphijos}{Toluca, Estado de México}
      {Líder de grupo y entrenamiento de staff}{Enero 2016 -- Julio 2022}
      % \resumeItemListStart
        % \resumeItem{Tensorflow}
          % {TensorFlow is an open source software library for numerical computation using data flow graphs; primarily used for training deep learning models. Worked on APIs and performance for training models on Tensor Processing Units (TPU).}
        % \resumeItem{Apache Beam}
          % {Apache Beam is a unified model for defining both batch and streaming data-parallel processing pipelines, as well as a set of language-specific SDKs for constructing pipelines and runners.}
      % \resumeItemListEnd
      
% --------Multiple Positions Heading------------
%    \resumeSubSubheading
%     {Software Engineer I}{Oct 2014 - Sep 2016}
%     \resumeItemListStart
%        \resumeItem{Apache Beam}
%          {Apache Beam is a unified model for defining both batch and streaming data-parallel processing pipelines}
%     \resumeItemListEnd
%    \resumeSubHeadingListEnd
%-------------------------------------------

    % \resumeSubheading
      % {Coursera}{Mountain View, CA}
      % {Senior Software Engineer}{Jan 2014 - Oct 2016}
      % \resumeItemListStart
        % \resumeItem{Notifications}
          % {Service for sending email, push and in-app notifications. Involved in features such as delivery time optimization, tracking, queuing and A/B testing. Built an internal app to run batch campaigns for marketing etc.}
        % \resumeItem{Nostos}
          % {Bulk data processing and injection service from Hadoop to Cassandra and provides a thin REST layer on top for serving offline computed data online.}
        % \resumeItem{Workflows}
          % {Dataduct an open source workflow framework to create and manage data pipelines leveraging reusables patterns to expedite developer productivity.}
        % \resumeItem{Data Collection}
          % {Designed the internal survey and crowd sourcing platform which allowed for creating various tasks for crowd sourcing or embedding surveys across the Coursera platform.}
        % \resumeItem{Dev Environment}
          % {Analytics environment based on docker and AWS, standardized the python and R dependencies. Wrote the core libraries that are shared by all data scientists.}
        % \resumeItem{Data Warehousing}
          % {Setup, schema design and management of Amazon Redshift. Built an internal app for access to the data using a web interface. Dataduct integration for daily ETL injection into Redshift.}
        % \resumeItem{Recommendations}
          % {Core service for all recommendation systems at Coursera, currently used on the homepage and throughout the content discovery process. Worked on both offline training and online serving.}
        % \resumeItem{Content Discovery}
          % {Improved content discovery by building a new onboarding experience on coursera. Using this to personalize the search and browse experience. Also worked on ranking and indexing improvements.}
        % \resumeItem{Course Dashboards}
          % {Instructor dashboards and learner surveying tools, which helped instructors run their class better by providing data on Assignments and Learner Activity.}
      % \resumeItemListEnd

    % \resumeSubheading
      % {Lucena Research}{Atlanta, GA}
      % {Data Scientist}{Summer 2012 and 2013}
      % \resumeItemListStart
        % \resumeItem{Portfolio Management}
          % {Created models for portfolio hedging,  portfolio optimization and price forecasting. Also creating a strategy backtesting engine used for simulating and backtesting strategies.}
        % \resumeItem{QuantDesk}
          % {Python backend for a web application used by hedge fund managers for portfolio management.}
      % \resumeItemListEnd

    % \resumeSubheading
      % {Georgia Institute of Technology}{Atlanta, GA}
      % {Research and Teaching Assistant}{Jan 2012 - Dec 2013}
      % \resumeItemListStart
        % \resumeItem{Research Assistant - Machine Learning}
          % {Research on machine learning for portfolio hedging and replication algorithms. Modeling low-risk \& continuous-return strategies. Developed the python library QSTK.}
        % \resumeItem{Teaching Assistant - Computational Investing}
          % {The online course on Coursera, had more than 100,000 students enrolled. It was featured on the 11 Alive News and the Atlanta Journal Constitution. Involved in creating assignment, exams and conducting recitation sessions. Also taught the on-campus version of the course.}
 % \resumeItemListEnd

\resumeSubHeadingListEnd


%-----------PROJECTS-----------------
\section{Proyectos}
  \resumeSubHeadingListStart
    % \resumeSubItem{Sly Proxy}
      % {Multi target, dynamic TCP proxy using Rust + Tokio for the back end, listening on a single port and
      % choosing from a multitude of targets thanks to eBPF programming.}
    \resumeSubItem{Sistema Gestor de Horarios}
      {Aplicación web que permite a los estudiantes crear un horario que
      se adapte a sus necesidades. Actualmente en beta abierto, con una
      publicación en mi
      \href{https://ccarral.github.io/en/projects/schedule-creator/}{blog}. Desarrollado en rust, embebido en el
      navegador con Vue + WebAssembly.}
    \resumeSubItem{\href{https://ccarral.github.io/en}{Cosmic Cube}}
      {Mi blog personal, en donde escribo sobre computación. Desarrollado en Vue.}
    \resumeSubItem{Tank}
      {Lenguaje de programación didáctico que controla un tanque en una
      rejilla cuadrada 12 x 12 con un intérprete desarrollado en rust,
      potenciando un \href{https://github.com/ccarral/tank-engine}{motor de juego} embebido en el navegador
      via Vue + WebAssembly.}
    \resumeSubItem{\href{https://github.com/ccarral/c-sudoku}{c-sudoku}}
      {Solucionador de sudokus, escrito en C.}
    \resumeSubItem{\href{https://github.com/ccarral/ensamblador8086}{ensamblador8086}}
      {Ensamblador x86, desarrollado en Java utilizando antlr4.}
    \resumeSubItem{mini6502}
      {Sencillo emulador del procesador 6502.}
  \resumeSubHeadingListEnd


% --------PROGRAMMING SKILLS------------
\section{Habilidades técnicas}
 \resumeSubHeadingListStart
  \resumeSubItem{Java Back End}{Desarrollo de aplicaciones enfocadas en
  microservicios utilizando Java + Spring Boot + Apache.}
  \resumeSubItem{Full stack Vue}{Desarrollo de aplicaciones modernas con
  Vue 3 y Bootstrap.}
  \resumeSubItem{Linux}{Más de tres años de experiencia en el uso de
  herramientas de sysadmin + conocimiento del funcionamiento interno del
  kernel.}
  \resumeSubItem{Desarrollo de aplicaciones en rust}{Desarrollo de
  aplicaciones asíncronas de bajo nivel.}
  \resumeSubItem{Scripting}{Automatización de tareas con python + bash.}
  \resumeSubItem{Git + Github}{Control de versiones con git y github.}
  \resumeSubItem{Bases de datos}{Extenso conocimiento de MySQL, Oracle y
  bases de datos no relacionales (nosql).}
  \resumeSubItem{CI/CD}{Experiencia en integración contínua utilizando
  herramientas como Github Actions y Travis CI.}
\resumeSubHeadingListEnd

% --------CERTIFICATIONS------------
\section{Certificaciones}
 \resumeSubHeadingListStart
  \resumeSubItem{Inglés Avanzado}{Certificación Cambridge B1 + TOEFL
  Avanzado.}
  \resumeSubItem{Scrum Fundamentals Certified (SFC)}{
  \href{https://www.scrumstudy.com/certification/verify?type=SFC&number=931455}{SCRUMstudy - Accreditation Body for Scrum and Agile}.}
  \resumeSubItem{Advanced C
  Programming}{\href{https://ude.my/UC-0dadf9b3-11ae-4f4d-8a76-6f1aa45b2b53/}{Certificado Udemy
  2022}.}
 \resumeSubHeadingListEnd
%-------------------------------------------

\section{Soft skills}
 \resumeSubHeadingListStart
  \resumeSubItem{Rápido aprendizaje}{Rápida absorción de contenido
  técnico (código, documentación, libros, RFC's, papers, etc.)}
  \resumeSubItem{Comunicación}{Clara comunicación oral y escrita, con
  facilidad de explicación de términos complejos a colegas sin entorno
  técnico.}
  \resumeSubItem{Líder de equipo}{Empujando a un equipo técnico en la dirección
  correcta, resolviendo puntos de fricción y detectando habilidades
  individuales.}
 \resumeSubHeadingListEnd



%-------------------------------------------
\end{document}
