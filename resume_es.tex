% Author : Sourabh Bajaj
% License : MIT
%------------------------

\documentclass[letterpaper,11pt]{article}

\usepackage{latexsym}
\usepackage[empty]{fullpage}
\usepackage{titlesec}
\usepackage{marvosym}
\usepackage[usenames,dvipsnames]{color}
\usepackage{verbatim}
\usepackage{enumitem}
% \usepackage[hidelinks]{hyperref}
% \usepackage{hyperref}
\usepackage[hidelinks,colorlinks=false,urlcolor=red]{hyperref}
\usepackage{fancyhdr}
\usepackage[spanish]{babel} % Cambiado a español
\usepackage{tabularx}
\input{glyphtounicode}

% \hypersetup{
    % colorlinks = true
    % urlcolor ={blue}
% }

\pagestyle{fancy}
\fancyhf{} % Limpiar todos los campos de encabezado y pie de página
\fancyfoot{}
\renewcommand{\headrulewidth}{0pt}
\renewcommand{\footrulewidth}{0pt}

% Ajustar márgenes
\addtolength{\oddsidemargin}{-0.5in}
\addtolength{\evensidemargin}{-0.5in}
\addtolength{\textwidth}{1in}
\addtolength{\topmargin}{-.5in}
\addtolength{\textheight}{1.0in}

% \urlstyle{same}

\raggedbottom
\raggedright
\setlength{\tabcolsep}{0in}

% Formato de secciones
\titleformat{\section}{
  \vspace{-4pt}\scshape\raggedright\large
}{}{0em}{}[\color{black}\titlerule \vspace{-5pt}]

% Asegurar que el PDF generado sea legible por máquinas y analizable por ATS
\pdfgentounicode=1

%-------------------------
% Comandos personalizados
\newcommand{\resumeItem}[2]{
  \item\small{
    \textbf{#1}{: #2 \vspace{-2pt}}
  }
}

\newcommand{\resumeItemAtom}[1]{
  \item\small{#1}
}

% En caso de que alguien necesite un encabezado que no deba estar en una lista
\newcommand{\resumeHeading}[4]{
    \begin{tabular*}{0.99\textwidth}[t]{l@{\extracolsep{\fill}}r}
      \textbf{#1} & #2 \\
      \textit{\small#3} & \textit{\small #4} \\
    \end{tabular*}\vspace{-5pt}
}

\newcommand{\resumeSubheading}[4]{
  \vspace{-1pt}\item
    \begin{tabular*}{0.97\textwidth}[t]{l@{\extracolsep{\fill}}r}
      \textbf{#1} & #2 \\
      \textit{\small#3} & \textit{\small #4} \\
    \end{tabular*}\vspace{-5pt}
}

\newcommand{\resumeSubSubheading}[2]{
    \begin{tabular*}{0.97\textwidth}{l@{\extracolsep{\fill}}r}
      \textit{\small#1} & \textit{\small #2} \\
    \end{tabular*}\vspace{-5pt}
}

\newcommand{\resumeSubItem}[2]{\resumeItem{#1}{#2}\vspace{-4pt}}
\newcommand{\resumeSubItemAtom}[1]{\resumeItemAtom{#1}\vspace{-4pt}}

\renewcommand{\labelitemii}{$\circ$}

\newcommand{\resumeSubHeadingListStart}{\begin{itemize}[leftmargin=*]}
\newcommand{\resumeSubHeadingListEnd}{\end{itemize}}
\newcommand{\resumeItemListStart}{\begin{itemize}}
\newcommand{\resumeItemListEnd}{\end{itemize}\vspace{-5pt}}

%-------------------------------------------
%%%%%%  CV COMIENZA AQUÍ  %%%%%%%%%%%%%%%%%%%%%%%%%%%%

\begin{document}

%----------ENCABEZADO-----------------
\begin{tabular*}{\textwidth}{l@{\extracolsep{\fill}}r}
	\textbf{\href{https://www.linkedin.com/in/carlos-carral-b668371a2}{\Large Carlos Carral Cortés - Ingeniero de Software}} & \href{mailto:carloscarral13@gmail.com}{carloscarral13@gmail.com}\\
	\href{https://ccarral.github.io}{https://ccarral.github.io} & +52-722-538-7535 \\
	& Ubicación: México
\end{tabular*}

% \vspace{20pt}
% \section{Perfil}
% Desarrollador con experiencia en herramientas populares de desarrollo front-end y back-end como python, java y javascript frameworks.
% Experimentado en flujos de desarrollo avanzados con Docker/Kubernetes con Redhat Quay.
% Experimentado en desarrollo embebido con C y rust.
% Flujos de desarrollo avanzados con git, github y gitlab.

%-----------EXPERIENCIA-----------------
\section{Experiencia Profesional}
\resumeSubHeadingListStart
\resumeSubheading{Vauxoo}{Remoto, México}
{Desarrollador Python}{Sep 2022 - Abr 2024}
\resumeItemListStart
% \resumeItem{Desarrollo en Python y javascript}
% {Desarrollo de un motor de programación basado en rust que asigna recursos de tiempo y condiciones fisicas a las ubicaciones.}
% \resumeItem{Programación asíncrona}
% {Programación asíncrona utilizando el tiempo de ejecución Tokio para aplicaciones de producción.}
\resumeItemAtom{Integré exitosamente APIs REST externas e internas junto con procedimientos xmlrpc.}
% \resumeItemAtom{Desarrollo y entrega de software que maneje datos sin procesar CSV y XLS.}
\resumeItemAtom{Diseñe e implementé de más de 30+ nuevas funciones en backend y frontend con python y Typescript.}
\resumeItemAtom{Impulsé la migración de varios sistemas legacy críticos, responsables de más de 300+ transacciones diarias.}
\resumeItemAtom{Incrementé la velocidad de generación de reportes para subsistemas críticos en un 400\%.}
\resumeItemAtom{Registré, reproduje, rastreé y corregí más de 60+ errores de producción.}
\resumeItemAtom{Depuré configuraciones de CI defectuosas y configuraciones de Docker de producción.}
\resumeItemAtom{Contribuí a software de código abierto a gran escala.}
\resumeItemAtom{Me reuní con stakeholders para abordar requerimientos del proyecto, plazos y obstáculos a eliminar.}
\resumeItemListEnd

\resumeSubheading{UAEMex, Departamento de Planificación y Desarrollo}{Toluca, Estado de México}
{DBA e Ingeniero de Automatización}{Agosto 2020 - Marzo 2021}
\resumeSubHeadingListEnd

%-----------EDUCACIÓN-----------------
\section{Educación}
\resumeSubHeadingListStart
\resumeSubheading
{Universidad Autónoma del Estado de México}{2023}
{Licenciatura en Ingeniería en Computación}{Toluca, Estado de México}
    \resumeItemListStart
    \resumeItemAtom{Obtuve el "Premio Ceneval al Desempeño de Excelencia EGEL", \href{https://reconocimiento.ceneval.edu.mx/busqueda-de-reconocimientos-2/?resultId=37772}{premio a la excelencia académica para los 2\% mejores calificados del EGEL.}}
    \resumeItemListEnd
\resumeSubheading
{Tecnológico de Monterrey}{}
{Preparatoria}{Toluca, Estado de México}
\resumeSubHeadingListEnd
% --------HABILIDADES TÉCNICAS------------
\section{Habilidades Técnicas}
\resumeSubHeadingListStart
\resumeSubItem{Full Stack Vue3}{Diseño de sitios web modernos y reactivos con el stack completo de Vue 3 .}
\resumeSubItem{Desarrollo Backend}{Python, Rust, Java Spring Boot, node y otros.}
% \resumeSubItem{Desarrollo de pila completa}{Rust, Vue3, Python y Node}
% \resumeSubItem{Desarrollo embebido}{Desarrollo orientado a eventos en placas ESP y STM de 64 bits.}
    % \resumeItemListStart
    % \resumeItemAtom{esp-idf}
    % \resumeItemAtom{Arduino}
    % \resumeItemAtom{Micropython}
    % \resumeItemListEnd
\resumeSubItem{Bases de datos relacionales y NoSQL}{Postgresql, Oracle, MySQL, MongoDB.}
\resumeSubItemAtom{\textbf{VCS con Git y Github}}
% \resumeSubItem{WASM}{Implementación de backends totalmente integrados en el
	% DOM con una pila de Rust +
	% wasm-pack + Vue 3.}
\resumeSubItem{Habilidades UNIX avanzadas}{}
    \resumeItemListStart
    \resumeItem{Instalación}{Instalación física de medios y red de Solaris, RHEL, Ubuntu Server y otros.}
    \resumeItem{Gestión}{Gestión remota de servidores a través de ssh, vi, cron, systemd, etc.}
    \resumeItemListEnd
\resumeSubItem{Scripting}{Automatización de tareas con Python y Bash.}
\resumeSubItem{DevOps}{Virtualización de Docker en flujos de trabajo de AWS.}
\resumeSubHeadingListEnd

%-----------PROYECTOS-----------------
\section{Proyectos Relacionados y Documentos Técnicos}
\resumeSubHeadingListStart
\resumeSubItem{\href{https://github.com/ccarral/minuteman}{minuteman}}
{Reloj despertador esp32 embebido con eventos, construido sobre el framework\href{https://github.com/espressif/esp-idf}{esp-idf} basado en C.}
\resumeSubItem{\href{https://ccarral.github.io/schedit-client/?pools=https://ccarral.github.io/schedit-client/files/plantillaA.csv}{sched.it}}
{Motor de programación de horarios con backend rust y una aplicación web React.}
\resumeSubItem{\href{https://ccarral.github.io/en/projects/hello_world_kernel_module/}{Escribiendo
		un módulo de kernel de Linux simple con rust desde cero}}{Documento técnico sobre cómo escribir un módulo de kernel Linux utilizando el
	\href{https://github.com/Rust-for-Linux}{árbol del kernel Rust-For-Linux} como base.}
\resumeSubItem{\href{https://github.com/ccarral/cp_liburing}{copia cp}}
{Clon basado en Rust del comando cp de Unix, aprovechando primitivas io\_uring de bajo nivel.}
\resumeSubItem{\href{https://github.com/ccarral/tank-vue}{tanque}}
{Juego didáctico, basado en un intérprete de lenguaje de programación simple desarrollado en rust.}
% \resumeSubItem{\href{https://ccarral.github.io/en/projects/freertos_and_tasks/}{Divertirse con FreeRTOS y primitivas de sincronización}}
% {Análisis técnico sobre un verdadero bloqueo encontrado al programar con FreeRTOS.}
% \resumeSubItem{\href{https://github.com/ccarral/sly-proxy}{sly-proxy}}
% {Proxy TCP respaldado por eBPF escrito en rust asíncrono con Tokio + Tower}
% \resumeSubItem{\href{https://github.com/ccarral/c-sudoku}{c-sudoku}}
% {Solucionador de sudoku por retroceso desarrollado en C.}
% \resumeSubItem{Tanque}
% {Lenguaje de programación didáctico simple para manipular un tanque en un
    % cuadrícula de 12 x 12, con un
    % \href{https://github.com/ccarral/interprete-tanques}{intérprete} desarrollado en
    % rust, impulsando un \href{https://github.com/ccarral/tank-engine}{motor de juego} \href{https://github.com/ccarral/tank-vue}{integrado en el navegador} a través de WASM + Vue.}

% \resumeSubItem{\href{https://github.com/ccarral/mini6502}{mini6502}}
% {Biblioteca de emulación 6502 simple y ligera desarrollada en rust puro.}
% \resumeSubItem{\href{https://github.com/ccarral/ensamblador8086}{ensamblador8086}}
% {Mock ensamblador x86 de una pasada. Genera código para un subconjunto del conjunto de instrucciones x86.
	% Implementado con antlr4.}
\resumeSubHeadingListEnd

% --------OTROS------------
\section{Otros}
\resumeSubHeadingListStart
\resumeSubItem{Certificación Fundamentos de Scrum (SFC)}{
	\href{https://www.scrumstudy.com/certification/verify?type=SFC&number=931455}{SCRUMstudy - Organismo de Acreditación para Scrum y Ágil}.}
\resumeSubItem{Programación C Avanzada}{
	\href{https://ude.my/UC-0dadf9b3-11ae-4f4d-8a76-6f1aa45b2b53/}{Certificado Udemy 2022}.}
\resumeSubItem{Idiomas}{Español (nativo), Inglés (fluido).}
\resumeSubHeadingListEnd
%-------------------------------------------
\end{document}
